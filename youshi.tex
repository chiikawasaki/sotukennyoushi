% !TEX program = lualatex
% !TEX program = lualatex
%
%   卒論/修論中間発表要旨テンプレート
%
\documentclass[a4j]{ltjsarticle}
\usepackage{graphicx}
\usepackage{tascmac}
\usepackage{verbatim}
\usepackage{url}

\renewcommand{\refname}{\normalsize 参考文献}
\newcounter{seccnt}
\setcounter{seccnt}{1}
\newcommand{\usection}[1]{\ \newline{\bf\underline{\theseccnt\stepcounter{seccnt}. #1}\hspace{10pt}}}

% twocolumn.sty  27 Jan 85
\twocolumn
\sloppy
\flushbottom
\parindent 1em
\leftmargini 2em
\leftmarginv .5em
\leftmarginvi .5em
\oddsidemargin 30pt 
\evensidemargin 30pt
\marginparwidth 48pt 
\marginparsep 10pt 
\textwidth 410pt 



\begin{document}
\topmargin -2.0cm
\textheight 25cm
\oddsidemargin  -8mm
\evensidemargin -8mm
\textwidth 18cm

\twocolumn[
\begin{center}
% --- 卒論タイトル (タイトルが2行に渡る場合) ---
%
%{\Large \bf 命令レベル並列プロセッサ用バリア型フェッチ機構\\}
%におけるNOP削減方式\\}
%\vspace{-0.1cm}
%{\Large \bf におけるNOP削減方式\\}
%{\small Decreasing of NOP Instruction on the Barrier \\}
%{\small Type Instruction Fetch Mechanism\\}
%\vspace{0.1cm}

% --- 卒論タイトル (タイトルが1行の場合) ---
%
{\Large \bf 生成AIによるAR仮想ペット自律行動の制御手法\\}
{\small A Control Method for Augmented Reality Pet Autonomous Behavior with Generative AI\\}
\vspace{0.1cm}


% --- 著者名(もし著者名(日本語と英語)が左右にずれるならhspaceを調整)する ---  
%
{\normalsize S225084 \hspace{0.5cm}泉二 咲希\\}
\vspace{-0.2cm}
{\small \hspace{0.5cm}Motoji Saki
}

\end{center}
\vspace{0.2cm} ]

\baselineskip 0.45cm
\thispagestyle{empty}

% -----   ここから先は、本文 -------------------------------- 

\usection{はじめに}
近年、AIを用いたペットロボットが普及しており、それらは人々に癒しを提供している\cite{1}。しかし、物理的な実体を持つロボットを動かすためには複雑な内部構造や高価な部品が必要となるため、一般家庭用ではぎごちない動きになるか、非常に高価・大型になってしまう。
また、使用していない時でも部屋のスペースを占有してしまうという問題もある。

これらの物理的な制約を解消する手段として、AR技術が有効である。AR技術を用いれば、物理的な制限に縛られず、滑らかで自然に動くペットを、物理的な場所を取らずにユーザーの目の前に提示できる。

しかし、従来のARペットアプリの多くは、あらかじめ決められたアニメーションを再生するだけに留まっているものが多い。 挙動が単純なパターンの繰り返しであれば、ユーザーはそこに生き物らしさを感じられず、癒やし効果が限られてしまう懸念がある。

そこで本研究では、生成AIを用いてその時の状況に応じた判断ができる自律機能を備えたARのペットを開発する。 具体的には、ユーザーの指示や環境に応じて自律的に行動を選択・学習する機能を実装する。
これにより、物理ロボットの物理的な制約と、従来のARペットの挙動の単純さという双方の課題を解決し、新しい形の癒やしに繋げることを目指す。


\usection{関連技術}
AR環境下での仮想ペットとのインタラクション体験を実現するために、Meta社製のスタンドアローン型MR/VRデバイスであるMeta Quest 3を使用する。Meta Quest 3は軽量で高性能なARデバイスであり、空間認識やハンドトラッキングなどの機能を備えることで、現実空間との高い融合度を実現している。

開発環境には、Unity Technologies社が提供するクロスプラットフォーム対応のゲームエンジンUnityを用いた。UnityはMeta Quest 3との互換性が高く、公式SDKの提供により、仮想ペットの動作制御やユーザーとのリアルタイムインタラクションの実装に適している。本研究では、仮想ペットと3D空間におけるユーザーとのインタラクション処理をすべてUnity上で構築する。

仮想ペットのモデルおよび動作には、Unity Asset Storeで提供されている「DOG Full Animations - Welsh Corgi」\cite{2}を採用した。本アセットには、犬の3Dモデルに加え、323種類のアニメーションが含まれている。これらを活用することで、待機、移動などの犬の自然な振る舞いを再現した。

仮想ペットの知能および応答生成には、Google社が提供する生成AIモデル「Gemini 2.0 Flash」のAPIを使用した。Gemini 2.0 Flashは、高い推論能力を持ちながら低遅延で動作する特徴を持つ。これにより、ユーザーの問いかけやアクションに対して、リアルタイム性を損なうことなく即座に反応を返すインタラクションシステムを実現している。

\usection{設計と実装}
アプリケーションは、主に「行動選択機能」「ボール遊び機能」「しつけ機能」の3つの機能により構成される。以下に各機能の詳細を述べる。

行動選択機能は、生成AIを用いて仮想ペットの行動を自律的に決定するものである。
仮想ペットは、待機、2種類の歩行、走行、座るという合計5つの行動パターンを有する。
行動決定の指標として、ペットの内部状態に $0.0$ 以上 $1.0$ 以下の値をとるパラメータ「エネルギー(Energy)」を定義した。システムは、このエネルギー値と直前の行動、および現在の行動をGemini APIに入力として送信する。
これにより、AIは現在のエネルギー状態に基づいた最適な行動を選択する。具体的には、エネルギー値が高い状態では歩行や走行といった活発な動作が選択されやすく、逆に低い状態では座るや待機といった休息行動の選択比率が高まるよう設計されている。
AIにより選択された行動が実行された後、その行動内容に応じてエネルギー値を更新する。
具体的には、歩く・走るなどの運動負荷の高い行動ではエネルギーを減少させ、逆に座る・待機などの休息行動では増加させる仕様とした。
こうして更新されたエネルギー値と直前の行動および現在の行動を再びAIへの入力として送信する。このプロセスを繰り返すことで、一貫性がありながらも予測不可能な、仮想ペットの自律的な行動選択を実現している。

ボール遊び機能は、ハンドトラッキングと物理演算を利用した、直感的な動作でボールを投げて遊ぶことができる機能である。 
ユーザーがAR空間に表示されたボールを自分の手で掴んで投げると、ボールは重力に従って放物線を描き落下する。犬は落下地点までボールを追いかけ、口にくわえてユーザーの元へ持ってくる。 ユーザーは戻ってきたボールを受け取り、何度でも繰り返し投げることができる。 
なお、現時点では犬は必ずボールを取りに行く仕様であるが、将来的には前述の「エネルギー」や、ユーザーとの「親密度」といったパラメータと連携させる設計としている。
これにより、犬の疲れ具合や信頼関係によって、ボールを取りに行くかどうかを自律的に判断させることを目指している。


\usection{評価}
現在、インタラクションシナリオにおいてハンドトラッキングによるボールの投球動作は実装が完了しており、今後は犬がそれを追いかけてユーザの元にボールを持ち帰る一連の行動を実現する予定である。

次に、より身体的な接触を取り入れたインタラクションとして、撫でる動作を検出し、それに応じて仮想ペットが好意的なアニメーション反応を示す機能を追加する。これは、動物との接触が視覚のみよりも高い癒し効果をもたらすとする先行研究\cite{2}に基づくものであり、触覚インタラクションを再現するために、実物のぬいぐるみにセンサーを搭載する計画である。
センサーにはBLE通信が可能なものの使用を検討しており、撫でる強さや範囲を検出することで、仮想ペットの表情や動作にリアルタイムで反映させる。

幸福感の評価指標として、心理的指標について以下の複数の尺度を候補として検討中である。

\begin{itemize}
    \item PANAS:ポジティブ感情とネガティブ感情を分けて評価し、幸福感の向上をポジティブ感情の増加とみなす。
    \item Subjective Happiness Scale (SHS):主観的幸福感を直接測定する短い質問紙である。
    \item WHO-5 Well-Being Index:心理的ウェルビーイングを簡便に測定でき、参加者の負担が少ない。
    \item POMS:一時的な感情状態を包括的に評価し、特にストレスやネガティブ感情の微小変化を得る。
\end{itemize}

今後は、実験の目的や参加者の負担を考慮しつつ、これらの尺度の中から最適なものを選定する。
生理指標としては、心拍を計測し、副交感神経活動の指標であるRMSSDを算出する。心拍は心拍センサーを用いて常時計測され、ストレス軽減の客観的な指標として用いる。


\usection{まとめ}
本研究では、AR技術を活用して仮想ペットとのインタラクション体験を構築し、ストレスに関する数値を下げ、幸福感の向上をもたらすことを検証することを目的とした。住宅事情や健康上の制約から実際のペットを飼うことが難しい人々に対して、代替手段としての仮想ペット体験を提供する意義は大きい。

開発面では、Meta Quest 3およびUnityを用いて、現実空間と仮想ペットとのインタラクションを実現した。今後は、撫でる動作をセンシングしてAR空間の犬の反応と連動させるなど、より没入感の高い触覚体験の実装を目指す。評価設計としては、主観的評価には複数の心理尺度を候補として検討しており、最終的に適切な尺度を選定する予定である。また、心拍を用いてストレスの変化を定量的に測定する。

これにより、AR技術がストレス緩和や幸福感の向上に果たし得る新しい役割を示すとともに、医療・福祉・教育分野への応用展開の可能性を探る基盤となることが期待される。

\small
\begin{thebibliography}{9}
\bibitem{1} Shuhei Imamura, Shota Okabe, Mana Toyama, Narumi Takano, Noriko Togashi, Kaname Hayashi, Kentaro Kajiya, and Takefumi Kikusui, 
\textit{Higher oxytocin concentrations occur in subjects who build affiliative relationships with companion robots}, 
    iScience, vol. 26, no. 12, 108562, 2023.
\texttt{https://doi.org/10.1016/j.isci.2023.108562}
\bibitem{2}
\textit{DOG Full Animations - Welsh Corgi},
\texttt{https://assetstore.unity.com/packages/3d/animations/dog-full-animations-welsh-corgi-247648}
\bibitem{3} Miho Nagasawa, Shouhei Mitsui, Shiori En, Nobuyo Ohtani, Mitsuaki Ohta, Yasuo Sakuma, Tatsushi Onaka, Kazutaka Mogi, and Takefumi Kikusui, 
\textit{Oxytocin-gaze positive loop and the coevolution of human–dog bonds}, 
Science, vol. 348, no. 6232, pp. 333-336, 2015.
\texttt{https://doi.org/10.1126/science.1261022}
\bibitem{4} Patricia Pendry and Jaymie L. Vandagriff, 
\textit{Animal Visitation Program (AVP) Reduces Cortisol Levels of University Students: A Randomized Controlled Trial}, 
AERA Open, vol. 5, no. 2, pp. 1–12, 2019.
\texttt{https://doi.org/10.1177/2332858419852592}

\end{thebibliography}
\end{document}
