% !TEX program = lualatex
% !TEX program = lualatex
%
%   卒論/修論中間発表要旨テンプレート
%
\documentclass[a4j]{ltjsarticle}
\usepackage{graphicx}
\usepackage{tascmac}
\usepackage{verbatim}
\usepackage{url}

\renewcommand{\refname}{\normalsize 参考文献}
\newcounter{seccnt}
\setcounter{seccnt}{1}
\newcommand{\usection}[1]{\ \newline{\bf\underline{\theseccnt\stepcounter{seccnt}. #1}\hspace{10pt}}}

% twocolumn.sty  27 Jan 85
\twocolumn
\sloppy
\flushbottom
\parindent 1em
\leftmargini 2em
\leftmarginv .5em
\leftmarginvi .5em
\oddsidemargin 30pt 
\evensidemargin 30pt
\marginparwidth 48pt 
\marginparsep 10pt 
\textwidth 410pt 



\begin{document}
\topmargin -2.0cm
\textheight 25cm
\oddsidemargin  -8mm
\evensidemargin -8mm
\textwidth 18cm

\twocolumn[
\begin{center}
% --- 卒論タイトル (タイトルが2行に渡る場合) ---
%
%{\Large \bf 命令レベル並列プロセッサ用バリア型フェッチ機構\\}
%におけるNOP削減方式\\}
%\vspace{-0.1cm}
%{\Large \bf におけるNOP削減方式\\}
%{\small Decreasing of NOP Instruction on the Barrier \\}
%{\small Type Instruction Fetch Mechanism\\}
%\vspace{0.1cm}

% --- 卒論タイトル (タイトルが1行の場合) ---
%
{\Large \bf 生成AIによるAR仮想ペット自律行動の制御手法\\}
{\small A Control Method for Augmented Reality Pet Autonomous Behavior with Generative AI\\}
\vspace{0.1cm}


% --- 著者名(もし著者名(日本語と英語)が左右にずれるならhspaceを調整)する ---  
%
{\normalsize S225084 \hspace{0.5cm}泉二 咲希\\}
\vspace{-0.2cm}
{\small \hspace{0.5cm}Motoji Saki
}

\end{center}
\vspace{0.2cm} ]

\baselineskip 0.45cm
\thispagestyle{empty}

% -----   ここから先は、本文 -------------------------------- 

\usection{はじめに}
近年、AIを用いたペットロボットが普及しており、それらは人々に癒しを提供している。しかし、物理的な実体を持つロボットを動かすためには複雑な内部構造や高価な部品が必要となるため、一般家庭用ではぎこちない動きになるか、非常に高価・大型になってしまう。
また、使用していない時でも部屋のスペースを占有してしまうという問題もある。

これらの物理的な制約を解消する手段として、AR技術が有効である。AR技術を用いれば、物理的な制限に縛られず、滑らかで自然に動くペットを、物理的な場所を取らずにユーザーの目の前に提示できる。

しかし、従来のARペットアプリの多くは、あらかじめ決められたアニメーションを再生するだけに留まっているものが多い。 挙動が単純なパターンの繰り返しであれば、ユーザーはそこに生き物らしさを感じられず、癒やし効果が限られてしまう懸念がある。

そこで本研究では、生成AIを用いてその時の状況に応じた判断ができる自律機能を備えたARのペットを開発する。 具体的には、ユーザーの指示や環境に応じて自律的に行動を選択・学習する機能を実装する。
これにより、物理ロボットの実体を持つがゆえの制約と、従来のARペットの挙動の単純さという双方の課題を解決し、新しい形の癒やしに繋げることを目指す。


\usection{関連技術}
AR環境下での仮想ペットとのインタラクション体験を実現するために、Meta社製のスタンドアローン型MR/VRデバイスであるMeta Quest 3を使用する。Meta Quest 3は軽量で高性能なARデバイスであり、空間認識やハンドトラッキングなどの機能を備えることで、現実空間との高い融合度を実現している。

開発環境には、Unity Technologies社が提供するクロスプラットフォーム対応のゲームエンジンUnityを用いた。UnityはMeta Quest 3との互換性が高く、公式SDKの提供により、仮想ペットの動作制御やユーザーとのリアルタイムインタラクションの実装に適している。本研究では、仮想ペットと3D空間におけるユーザーとのインタラクション処理をすべてUnity上で構築する。

仮想ペットのモデルおよび動作には、Unity Asset Storeで提供されている「DOG Full Animations - Welsh Corgi」\cite{1}を採用した。本アセットには、犬の3Dモデルに加え、323種類のアニメーションが含まれている。これらを活用することで、待機、移動などの犬の自然な振る舞いを再現した。

仮想ペットの知能および応答生成には、Google社が提供する生成AIモデル「Gemini 2.0 Flash」のAPIを使用した。Gemini 2.0 Flashは、高い推論能力を持ちながら低遅延で動作する特徴を持つ。これにより、ユーザーの問いかけやアクションに対して、リアルタイム性を損なうことなく即座に反応を返すインタラクションシステムを実現している。

\usection{設計と実装}
アプリケーションは、主に「行動選択機能」「ボール遊び機能」「しつけ機能」の3つの機能により構成される。以下に各機能の詳細を述べる。

行動選択機能は、生成AIを用いて仮想ペットの行動を自律的に決定するものである。
仮想ペットは、待機、2種類の歩行、走行、座るという合計5つの行動パターンを有する。
行動決定の指標として、ペットの内部状態に $0.0$ 以上 $1.0$ 以下の値をとるパラメータ「エネルギー」を定義した。システムは、このエネルギー値と直前の行動、および現在の行動をGemini APIに入力として送信する。
これにより、AIは現在のエネルギー状態に基づいた最適な行動を選択する。具体的には、エネルギー値が高い状態では歩行や走行といった活発な動作が選択されやすく、逆に低い状態では座るや待機といった休息行動の選択比率が高まるよう設計されている。
AIにより選択された行動が実行された後、その行動内容に応じてエネルギー値を更新する。
具体的には、歩く・走るなどの運動負荷の高い行動ではエネルギーを減少させ、逆に座る・待機などの休息行動では増加させる仕様とした。
こうして更新されたエネルギー値と直前の行動および現在の行動を再びAIへの入力として送信する。このプロセスを繰り返すことで、一貫性がありながらも予測不可能な、仮想ペットの自律的な行動選択を実現している。

ボール遊び機能は、ハンドトラッキングと物理演算を利用した、直感的な動作でボールを投げて遊ぶことができる機能である。 
ユーザーがAR空間に表示されたボールを自分の手で掴んで投げると、ボールは重力に従って放物線を描き落下する。犬は落下地点までボールを追いかけ、口にくわえてユーザーの元へ持ってくる。 ユーザーは戻ってきたボールを受け取り、何度でも繰り返し投げることができる。 
なお、現時点では犬は必ずボールを取りに行く仕様であるが、将来的には前述の「エネルギー」や、ユーザーとの「親密度」といったパラメータと連携させる設計としている。
これにより、犬の疲れ具合や信頼関係によって、ボールを取りに行くかどうかを自律的に判断させることを目指している。

最後に、「しつけ機能」について、この機能はハンドジェスチャを通じて犬に「おすわり」を学習させる機能である。
 犬の内部状態として、訓練の成果を表す$0.0$ 以上 $100.0$ 以下の値をとるパラメータ「熟練度」を定義した。
 まず、訓練モードでは、ユーザーが仮想のお菓子を持ち、手を下から上に上げる動作を行う。
 これに合わせて犬がおすわりを行うと、システムは訓練の実行情報と現在のエネルギー値をGemini APIに送信する。AIは入力されたエネルギー値に基づき、今回獲得する熟練度の上昇量を決定する。
 この際、エネルギー値が高いほど熟練度が上がりやすく、逆に低い状態では熟練度の上昇が抑制されるように設計されている。
 次に、お菓子を持たずに拳を下から上へ上げる動作を行うと、現在の熟練度とエネルギー値がGemini APIに送信される。AIはこれらの値を総合的に判断し、反応を決定する。
 AIからの応答はSit,Confused,Tiredの3種類あり、Sitでは素直におすわりを実行、Confusedでは首をかしげて命令が伝わらない様子を見せ、Tiredでは命令を拒む動作を行う設定である。
 これにより、熟練度や体調によって反応が変わる生物的な振る舞いを表現している。


\usection{評価}
しつけ機能の有効性を評価するため、熟練度とエネルギー値が成功率に与える影響を調査した。 
実験は、エネルギー値を「低(0.2)」、「中(0.5)」、「高(0.8)」の3段階に設定し、各条件下で計3回実施した。手順として、合計100回の訓練を行う過程で、10試行ごとに10回の本番テストを行い、その成功率を記録した。 

図\ref{fig:proficiency}は、各エネルギー条件下における訓練回数(横軸)と熟練度(縦軸)の関係を示している。
まず、エネルギー値が0.2の条件では、0.5および0.8の条件と比較して、熟練度の上昇が著しく抑制されていることが確認できる。これは、本システムに実装された「疲労時には命令に対する反応や学習意欲が低下する」というロジックが、生成AIの判断に正しく反映された結果であると考えられる。

一方で、エネルギー値が0.5と0.8の条件間では、最終的な獲得熟練度に大きな差は確認されなかった。このことから、エネルギーがある一定の閾値を超えている状態であれば、学習効率は飽和し、それ以上の過剰なエネルギーが学習速度に寄与しない仕様となっていることが示唆される。

なお、エネルギー0.5の条件において、試行回数40-50回付近で熟練度の急激な上昇が見られる。これは、システムの不具合ではなく、生成AIの持つランダム性によるゆらぎに起因するものと推察される。本システムでは、この予測不可能性もまた、生物特有の「気まぐれ」な性質の一部として許容している。

図\ref{fig:success_rate}は、各エネルギー条件下における訓練回数(横軸)と本番動作の成功率(縦軸)の関係を示している。

まず、エネルギー値が0.2の条件では、訓練回数を重ねても成功率は0\%のままであった。これは、前述の通り、低エネルギー状態により学習意欲が低下し、必要な熟練度を獲得できなかったことに起因すると考えられる。

次に、エネルギー値が0.5と0.8の条件を比較すると、獲得した熟練度には大きな差が見られなかったにも関わらず、成功率においてはエネルギー0.8の条件が最終的に高い数値を示した。
この結果は、AIの行動決定プロセスにおいて、エネルギー値が学習効率だけでなく実行意欲にも直接的な影響を与えていることを示唆している。
すなわち、十分な熟練度があっても、エネルギーに余裕がなければ、AIは確率的に命令を拒否する仕様となっており、エネルギー0.8の条件ではその拒否率が低減されたことで成功率が向上したと考えられる。

以上の結果より、本システムにおける犬の応答は、熟練度とエネルギーという二つの変数が複合的に影響を与えた結果であることが示唆された。
熟練度が高くてもエネルギーが不足していればパフォーマンスは低下し、逆にエネルギーが十分でも熟練度が伴わなければ成功には至らない。両パラメータが相互補完的に機能する設計としたことで、状況に応じた動的な挙動変化を実現できたと言える。
\begin{figure}[b]
    \centering
    \includegraphics[width=1.1\linewidth]{zyukurenndo.png}
    \caption{訓練回数に対する熟練度の推移}
    \label{fig:proficiency}
  \end{figure}
  \begin{figure}[b]
    \centering
    \includegraphics[width=1.1\linewidth]{my_graph.png}
    \caption{訓練回数に対する成功率の推移}
    \label{fig:success_rate}
  \end{figure}

\usection{まとめ}
本研究では、生成AIを用いて、内部状態に基づき自律的に行動決定を行うAR仮想ペットシステムの構築を行った。仮想ペットの機能として「行動選択機能」「ボール遊び機能」「しつけ機能」を開発した。

しつけ機能を用いた評価実験では、犬のおすわりの成功率が、熟練度とエネルギーの双方に依存して変化することを確認した。
特に、熟練度が高くてもエネルギーが低い場合には成功率が低下し、逆にエネルギーが十分でも熟練度が不足していれば成功に至らないという結果が得られた。
この結果は、提案手法における行動決定が単一のルールではなく、複数の変数が複合的に作用して決定されていることを示している。
以上より、本研究で提案したパラメータ制御と生成AIを組み合わせる手法は、状況に応じて柔軟に応答するシステムとして有効であることを確認した。

\small
\begin{thebibliography}{9}
\bibitem{1}
\textit{DOG Full Animations - Welsh Corgi},
\texttt{https://assetstore.unity.com/packages/3d/animations/dog-full-animations-welsh-corgi-247648}

\end{thebibliography}
\end{document}
